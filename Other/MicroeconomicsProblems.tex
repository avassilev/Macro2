% !TeX spellcheck = bg_BG
\documentclass[10pt,a4paper]{article}
\usepackage[utf8]{inputenc}
\usepackage[T1]{fontenc}
\usepackage[bulgarian]{babel}
\usepackage{amsmath}
\usepackage{amsfonts}
\usepackage{amssymb}
\title{Микроикономически задачи}
\date{}
\begin{document}
	\maketitle

\textbf{Задача 1.} В примера за спестяване и потребление между два периода разгледахме задачата за максимизиране на функцията на полезност \[ U(c_1,c_2) = \ln c_1 + \beta \ln c_2, \qquad \beta \in (0,1) \] при бюджетно ограничение \[ c_1 + \dfrac{c_2}{1+r} = y, \]
където $ c_i $ е потреблението в период $ i $, $ r $ е лихвеният процент, а $ y $ е доходът, получаван в първия период. Неявно допускане в задачата е, че цената на потребителската стока е една и съща между двата периода и можем да приемем, че $ p_1=p_2=1 $.

Модифицирайте задачата за случая на произволни цени $ p_1 $ и $ p_2 $. Изведете необходимите условия за оптималност и напишете решенията $ c^*_1 $ и $ c^*_2 $  за този случай.

\bigskip

\textbf{Задача 2.} Да разгледаме задачата за максимизиране на полезността от потреблението за три периода, когато функцията на полезност се задава от \[ U(c_1,c_2,c_3) = \ln c_1 + \beta \ln c_2 + \beta^2 \ln c_3, \qquad \beta \in (0,1). \]
Ако сме приели постоянни цени $ p_1 = p_2 = p_3 = 1 $, постоянен лихвен процент $ r $ и доход $ y $, получаван в първия период, бюджетното ограничение е \[ c_1 + \dfrac{c_2}{1+r} + \dfrac{c_3}{(1+r)^2} = y. \]

Модифицирайте задачата за случая, когато лихвеният процент се променя и е означен съответно с $ r_2 $ между периоди 1 и 2 и с $ r_3 $ между периоди 2 и 3. Как изглеждат необходимите условия за оптималност, след като се освободите от множителя на Лагранж? (Не е необходимо да решавате системата и да получите оптималните $ c^*_i $.)


\end{document}